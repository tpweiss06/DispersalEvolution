\documentclass[11pt]{article}
\usepackage[sc]{mathpazo} %Like Palatino with extensive math support
\usepackage{fullpage}
\usepackage[authoryear,sectionbib,sort]{natbib}
\linespread{1.7}
\usepackage[utf8]{inputenc}
\usepackage{lineno}
\usepackage{titlesec}
\titleformat{\section}[block]{\Large\bfseries\filcenter}{\thesection}{1em}{}
\titleformat{\subsection}[block]{\Large\itshape\filcenter}{\thesubsection}{1em}{}
\titleformat{\subsubsection}[block]{\large\itshape}{\thesubsubsection}{1em}{}
\titleformat{\paragraph}[runin]{\itshape}{\theparagraph}{1em}{}[. ]\renewcommand{\refname}{Literature Cited}

\usepackage{mathptmx}
\usepackage{hyperref}
\usepackage{geometry}
\usepackage[centertags]{amsmath}
\usepackage{amssymb}
\usepackage{amsthm}
\usepackage{fancybox}
\usepackage{graphicx}
\usepackage{graphics}
\newcommand{\s}{^{(s)}}
\usepackage{lineno}

%\title{Dispersal evolution during range expansions and shifts}
%\title{Genetic constraints on dispersal evolution via spatial sorting}
%\title{Genetic architecture and dispersal evolution via spatial sorting}
\title{Evolutionary constraints and ecological consequences of dispersal evolution in range shifts}

\author{Christopher Weiss-Lehman$^{1,\ast}$ \\ 
Allison K. Shaw$^{2}$}

\date{}

\begin{document}

\maketitle

\noindent{} 1. Department of Botany, University of Wyoming, Laramie, Wyoming 82071;

\noindent{} 2. Department of Ecology, Evolution, and Behavior, University of Minnesota, Saint Paul, Minnesota 55108;

\noindent{} $\ast$ Corresponding author; e-mail: cweissle@uwyo.edu

\bigskip

\textit{Keywords}: range shifts, spatial sorting, individual-based model, Allee effect, genetic architecture

\noindent{\footnotesize Prepared using the suggested \LaTeX{} template for \textit{Am.\ Nat.}}

\linenumbers{}
\modulolinenumbers[3]

\newpage{}

\section*{Abstract}
\newpage{}

\section*{Introduction}
Range expansions and shifts have become ubiquitous features of modern biomes. For centuries, humans have facilitated the range expansions of invasive species through travel, commerce, agriculture, and other routes~\citep{elton1958ecology}, a trend that has only increased with further globalization~\citep{hulme2009trade}. In recent decades, humans have further eroded traditional biogeographic boundaries through anthropogenic climate change, leading to range shifts among a wide variety of taxa~\citep{parmesan2006ecological}. Therefore, it is crucial to understand both the ecological and evolutionary mechanisms driving population dynamics in these range expansions and shifts. While range expansions have been studied from both ecological~\citep{skellam1951random} and evolutionary~\citep{fisher1937wave} contexts, recent research has demonstrated the crucial interplay of ecological and evolutionary mechanisms in driving population dynamics~\citep{shaw2015dispersal, weiss2017rapid, ochocki2017rapid, williams2016rapid} (Could also cite the Ecology letters paper here if it comes out in time). Conservation efforts to preserve biodiversity in the face of such widespread range shifts due to climate change will hinge, partly, on how well we can understand and predict their underlying eco-evolutionary dynamics.

One crucial feature of the eco-evolutionary dynamics of range expansions and shifts is the process of spatial sorting, in which individuals become sorted in space according to their dispersal phenotypes~\citep{shine2011evolutionary}. The expansion edge, by definition, will be composed of the individuals which dispersed the farthest distances, while shorter dispersal phenotypes will be closer to the population core. If dispersal is a heritable trait, then offspring produced at the edge will also have high dispersal phenotypes~\citep{fronhofer2015eco}, leading to a feedback between increased dispersal evolution at the edge and greater rates of expansion~\citep{phillips2006invasion, alex2013evolution, burton2010trade, williams2016rapid}. In addition to causing more rapid rates of expansion on average, some studies suggests that spatial sorting and other evolutionary processes can dramatically increase the variance in expansion rates among replicated populations as well~\citep{weiss2017rapid, ochocki2017rapid, phillips2015evolutionary}. In the context of climate driven range shifts, the role of dispersal evolution in driving expansion dynamics could be crucial to a species' ability to persist.

As the number of documented range shifts increase around the globe, researchers are documenting cases of spatial sorting leading to the evolution of increased dispersal in shifting species. In the United Kingdom, northern range margin populations of the speckled wood butterfly and two species of bush crickets have all shown increased dispersal capabilities relative to southern populations~\citep{hill1999flight, thomas2001ecological}. However, given the increasing rate of climate change~\citep{chen2017increasing} and the substantial gap between current dispersal capabilities and those necessary to keep pace with climate change in some species~\citep{schloss2012dispersal}, it is unclear if dispersal evolution will be enough to rescue faltering populations. Some theoretical models suggest that it could indeed provide a buffer, allowing populations otherwise doomed to extinction to persist~\citep{boeye2013more}, but others show it may be insufficient by itself to prevent the extinction of struggling populations~\citep{weiss2019spatial}. 

The key factor likely to determine the potential role of dispersal evolution in climate driven range shifts is the speed with which dispersal evolution occurs. If evolution of dispersal ability occurs too slowly relative to the speed of climate change it cannot rescue lagging populations. Importantly, theoretical models finding differing results for the role of dispersal evolution in climate driven range shifts have made different assumptions about the life history of the organisms and the nature of dispersal. Many of these different assumptions, including sexual vs. asexual reproduction, haploid vs. diploid genetics, and the number of loci encoding dispersal, have all been shown to affect the rate of evolution due to natural selection in other traits~\citep{orr1994does, zeyl2003evolutionary, goddard2005sex, pritchard2010genetics}. However, as dispersal evolution occurs primarily via the spatial sorting of individuals rather than via traditional natural selection~\citep{shine2011evolutionary}, it is unclear how strongly these features might influence the evolution of dispersal during range expansions and shifts. Additionally, some of these genetic assumptions can have impacts on the ecological dynamics of species as well. In particular, sexual reproduction can introduce complications such as mate finding Allee effects and differences among mating systems that also have the potential to alter the dynamics of range expansions and shifts~\citep{shaw2015dispersal}. Thus, it is important to better understand the role of these assumptions in the evolutionary and ecological dynamics of range expansions and shifts.

Here, we constructed an individual-based model to explore the role of genetic structure (ploidy level and the number of loci encoding dispersal) and the mode of reproduction (sexual vs. asexual and the role of self fertilization) in dispersal evolution in range expansions and shifts. By using a single, common framework to explore these factors, we directly compared the effect of each on the rate of dispersal evolution and related them to the extinction risk faced by populations shifting their ranges in response to climate change. We predicted that, analogous to traits evolving via natural selection, spatial sorting would more effectively increase dispersal phenotypes under sexual reproduction and with greater numbers of loci contributing to dispersal. Further, we hypothesized that this increased rate of dispersal evolution would lead to lower extinction risk in populations shifting their ranges in response to simulated climate change.

\section*{Model overview}

We performed simulations of relatively simple range expansions in which the environment was constant through time and space to better understand the impacts of these assumptions on dispersal evolution via spatial sorting

\subsection*{Purpose} 
Here, we use a common modeling framework to disentangle the role of different factors on the speed of dispersal evolution in moving populations. Specifically, we vary the number of loci defining the dispersal trait, the number of chromosome copies comprising the genome (haploid vs. diploid), and the mode of reproduction (asexual vs. sexual). For each scenario, we simulate population dynamics within a one dimensional stable range for $5000$ generations to reach an equilibrium distribution of genotypes. Then, we simulate a range expansion with individuals from the edge of the range to quantify the time it takes for the expanding populations to reach a threshold dispersal phenotype. Finally, we use the original population to simulate climate change with a speed equal to the speed of an expansion wave defined by the same threshold dispersal phenotype. We can therefore directly compare the time to evolution of the dispersal phenotype in the range expansion with the extinction probability of populations shifting their range.

\subsection*{State variables and scales} 
We track individuals defined by either one or two chromosome copies and a number of loci defining dispersal traits. The genetic structure and mode of reproduction are set at the beginning of each simulation. In sexually reproducing, dioecious populations, individuals are further delineated as male and female. Space is modeled as a linear, one dimensional array of discrete habitat patches. The carrying capacity of each patch varies and is used to define the characteristics of the population's range. For simplicity, there is no variation in intrinsic fitness among individuals.

\subsection*{Process overview and scheduling} 
We assume discrete, non-overlapping generations divided between two phases: dispersal and reproduction. Individuals first disperse from their natal patches with distance determined by their phenotypes and direction assigned randomly. After dispersal, individuals reproduce within their new patches, thus limiting the mating pool to local individuals when reproduction is sexual. The number of successful reproduction events (i.e. leading to a surviving offspring) in each patch is determined with a stochastic implementation of the classic Ricker model with parentage assigned randomly. Individuals may contribute to multiple successful reproduction events and no restrictions are placed on the number of mates for either males or females in dioecious populations (polygynandrous mating). Offspring inherit alleles from their parent(s) assuming independent segregation and a mutation process. After reproduction, all members of the current generation perish and their offspring disperse to begin the next generation.

\section*{Design concepts}
\subsection*{Emergence} 
Emergent phenomena include the spatial distributions of population abundance, genetic diversity, and dispersal phenotypes in the stable range at equilibrium, the evolutionary trajectories of dispersal in expanding and shifting populations, and the extinction probability of populations undergoing range shifts.

\subsection*{Stochasticity} 
Stochastic processes include reproduction, dispersal distance and direction, and inheritance of loci (described below). Environmental conditions are deterministic to remove any confounding influence of environmental stochasticity on the results.

\subsection*{Interactions} 
Individuals in all experimental scenarios interact via density dependent competition within patches. In scenarios with sexual reproduction and without obligate selfing, individuals also interact via mating within local patches. 

\subsection*{Desired output} 
After reaching a spatial equilibrium within a stable range, each simulation saves the full details of all individuals from the latest generation. Using that output, we then simulate two additional scenarios: a range expansion and a range shift. From the range expansion scenario, we save aggregate data on population abundance and the mean and variance of dispersal values within each patch in each generation. Additionally, save the generation at which the population at the edge of expansion reaches the threshold dispersal value. From the range shift, we similarly aggregate data on population abundance and the mean and variance of dispersal values from each patch in each generation. If the population does not go extinct before evolving the dispersal phenotype required to track climate change, the simulation will continue for an additional $50$ generations to ensure that the population survives, and then output the full details of all individuals from the latest generation.

\section*{Details}
\subsection*{Initialization} 
To begin each simulation, several patches in the center of the range are populated to full carrying capacity ($K_{max}$) with individuals whose genotypes are randomly generated from a normal distribution of allele frequencies. Simulations then continue for $5000$ generations within the stable range to reach a spatial equilibrium, after which range expansion and range shift simulations are performed. For the range expansion simulations, $K_{max}$ individuals are randomly chosen from the edge of the landscape and placed within a single patch of an empty, uniformly habitable landscape. Due to computational constraints, we track only the leading edge of the expansion rather than tracking every individual in the expanding population. For the range shift, the entire original population is used as the environmental conditions begin to shift in space to simulate climate change.

\subsection*{Submodels}
\paragraph{Defining the abiotic environment}
Environmental conditions in each patch are defined by the carrying capacity, $K(x)$. To simulate range boundaries, we assume that $K(x)$ is set to a constant, maximum value, $K_{max}$, throughout the core of the range (defined by $\tau$). At the range edges, $K(x)$ declines to $0$ at a constant proportion of $\gamma$ per patch. The range center is defined by $\beta$, which is held constant for the initial simulations of a stable range and then shifted linearly to simulate climate change (see below). More precisely, the carrying capacity of each patch is given by
\begin{equation}
K(x)=
\begin{cases}
	max(0, (1-\gamma(\beta(t) - \tau - x))K_{max}) & x < \beta(t) - \tau \\
	K_{max} & \beta(t) - \tau \leq x \leq \beta(t) + \tau \\
	max(0, (1-\gamma(x - \beta(t) - \tau))K_{max}) & x > \beta(t) + \tau
\end{cases}
\end{equation}
 
 To simulate climate change, $\beta(t)$ is varied linearly with time at rate $\nu$ (i.e. $\beta(t)=\nu t$). For simulations of range expansion, the environmental conditions are simplified to a homogenous landscape of patches all with carrying capacities of $K_{max}$. 

\paragraph{Local population dynamics}
Population growth within each patch is modeled as a stochastic implementation of the classic Ricker model~\citep{ricker1954stock, melbourne2008extinction}. Importantly, this equation can account for asexual reproduction or sexual reproduction with explicit males and females~\citep{melbourne2008extinction}. In the simpler cases of asexual reproduction and sexual reproduction of monoecious individuals, the expected population size in patch $x$ at time $t+1$ is given by
\begin{equation}
\hat{N}_{t+1,x}=N_{t,x}Re^{\frac{-RN_{t,x}}{K(x)}}
\end{equation}
where $N_{t,x}$ is the current population size of patch $x$, $R$ is the intrinsic growth rate, and $K(x)$ is the carrying capacity of patch $x$. When modeling population growth of sexually reproducing, dioecious individuals, the expected population size in patch $x$ at time $t+1$ becomes
\begin{equation}
\hat{N}_{t+1,x}=F_{t,x}\frac{R}{\psi}e^{\frac{-RN_{t,x}}{K(x)}}
\end{equation}
where $F_{t,x}$ is the number of females in patch $x$ at time $t$ and $\psi$ is the expected ratio of female offspring. To account for demographic stochasticity, the expected population size is then used to generate the realized population size in the next generation from a Poisson distribution.
\begin{equation}
N_{t+1,x}\sim Poisson(\hat{N}_{t+1,x})
\end{equation}
This realized population size in the next generation can also be thought of as the number of successful reproduction events in the current generation. For each of these successful reproduction events, parentage is assigned randomly, according to the mode of reproduction. For asexual reproduction, a single individual is drawn randomly and with replacement (thus allowing for multiple offspring from a single individual) from the local population for each successful reproduction event. For sexual reproduction events, parental pairs are formed according to the nature of individuals in the population (monoecious or dioecious) and the probability of self fertilization ($\omega$). In populations of monoecious individuals, a single parent is first randomly selected from the local population. With probability $\omega$ the individual self fertilizes and with probability $1-\omega$ a second individual is required. If no other individual is present in the patch, then the reproduction event fails. Thus, by varying the value of $\omega$, we can simulate populations ranging from obligate selfing to obligate outcrossing and the resultant mate finding Allee effect. In populations of dioecious individuals, each reproduction event must include both a male and female parent, imposing an even greater mate finding Allee effect in these populations. Males and females are selected randomly and with replacement from the population for each reproduction event and both males and females can mate multiple times (polygynandry). 

Once parentage is determined for each reproductive event, offspring inherit alleles from their parent(s), assuming no linkage among loci and a mutation process defined by two parameters: the per allele probability of mutation ($\nu_{m}$) and the standard deviation of mutational effects ($\sigma_{m}$). Thus, when a mutation occurs with probability $\nu_{m}$, the new allele value is drawn from a normal distribution with mean equal to the original allele value and a standard deviation of $\sigma_{m}$).

\paragraph{Dispersal}
Finally, individuals disperse according to an exponential dispersal kernel defined by each individual's dispersal phenotype. An individual's dispersal phenotype is the expected dispersal distance and is given by
\begin{equation}
d_{i} = \frac{\hat{d} e^{\rho\Sigma L}}{1+e^{\rho\Sigma L}} 
\end{equation}
where $\hat{d}$ is the maximum expected dispersal distance in terms of discrete patches, $\rho$ is a constant determining the slope of the transition between $0$ and $\hat{d}$, and the summation is taken across all alleles contributing to dispersal. Thus, loci are assumed to contribute additively with no dominance or epistasis. The expected dispersal distance, $d_{i}$ is then used to draw a realized distance from an exponential dispersal kernel and direction (forward or backward in the linear landscape) is determined randomly. Since the dispersal phenotype is the expected value of the exponential dispersal kernel, it can be used directly to calculate the two dimensional diffusion coefficient of population spread ($D$). Specifically, since $d_{i}^{2}$ represents the mean squared displacement of an individual with dispersal phenotype $d_{i}$, the one dimensional diffusion coefficient can be calculated as
\begin{equation}
D = \frac{1}{2}d_{i}^{2}
\end{equation}

\bibliographystyle{amnat}
\bibliography{main_bib}

\end{document}